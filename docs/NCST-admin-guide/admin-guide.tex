\documentclass[lang=cn,hazy,screen,blue,14pt]{elegantnote}

\title{华北理工大学数学建模协会竞赛管理软件\\——管理员使用文档}

\author{16 计科刘佳玮}
\institute{Elegant\LaTeX{} Program}
\version{1.0}
\date{\today}

\begin{document}
\maketitle
\newpage
\tableofcontents
\newpage

\section{数据库}

关于所有比赛的所获奖项,美赛用英文的outstanding winner;国赛的用国家一等奖,省级一等奖;其他的比赛统称:一等奖,二等奖,三等奖和成功参与奖,直接置入软件内部。

数据库是软件的灵魂,或者说只要数据库在,就能开发出更多的功能。在这里标注数据库中共有多少个表,每个表的含义和注意事项,这里是管理员实用软件需要注意的。

\begin{note}
主键是数据表的唯一标识符,同一个数据表内主键不能重复,用于唯一标识表内的一条数据。
\end{note}

\begin{note}
长度不能超过20指:字符串的长度要小于20个字符。
\end{note}

\begin{enumerate}
    \item 超级管理员:字段为id,账号和密码。区分超级管理员和普通管理员的方式为:超级管理员的账号以@admin结尾,如ljw@admin。且wuyuhang@admin的账号不能被删除。主键为id(系统自动设置),故账号名可重复(不建议这样,一个人一个账号)。账号的长度不能超过20,密码的长度小于100。
    \item 普通管理员:字段为id,账号和密码。账号不以@admin结尾即为普通管理员。主键为id(系统自动设置),账号长度不能超过20,密码长度不能超过100。
    \item 学院表:字段只有学院名。主键为学院名称,长度不能超过30。
    \item 专业表:字段为学院名和专业名。主键为专业名称,长度不能超过50。所以,矿业工程学院的采矿工程会和以升学院的采矿工程专业重名导致无法插入。所以为区分以升这个学院里面的专业,在插入以升专业的时候要标注学院:采矿工程(以升)。
    \item 教师表:字段为教师工号,学院和教师姓名。教师的工号为主键,长度小于10个字符;教师姓名小于20个字符;所属的学院必须从学院表中已有的学院进行选择,防止录入错误。
    \item 竞赛表:字段为比赛名称、主办方和比赛等级。比赛名称是主键,长度不能超过30个字符;主办方不能超过100个字符;等级不能超过25个字符。
    \item 博客表:字段为博客网址、学生姓名、个性签名、年级和最高奖项。其中网址是主键,长度不能超过200个字符;学生姓名不超过20个字符;个性签名在100个字符以内;年级不超过10个字符;获奖不超过50个字符;专业不超过50个字符。
    \item 学生表:字段包括id、学号、姓名、比赛、级别、教师、录入时间、参赛时间、获奖时间、学院、专业、联系方式和性别。其中id为主键(系统自动生成);学号长度不能超过20个字符;姓名长度不能超过10个字符;竞赛必须从已有的竞赛表中选择;比赛级别长度不能超过50个字符;默认为成功参赛奖;学院、专业必须从已有的数据表中进行选择;联系方式长度不能超过20个字符。考虑到会经常以学号和姓名对学生进行查询,所以对学号和姓名建立索引,加快查询速度。
\end{enumerate}

\begin{note}
我没有想好如何标明一场比赛中,谁是谁的队友;或者说如何标明一个人的队友,如果有更好的想法可以改进。
\end{note}

\section{信息管理}

若是超级管理员,开放信息管理界面。管理全部信息,包括:

\begin{enumerate}
\item 注册与删除普通管理员帐号,wuyuhang@admin不能被删除。
\item 学院专业管理。先插入学院,在插入专业,不能学院和专业同时插入;插入专业时必须填写所在学院,防止胡乱插入;对于专业改名,删除原专业重新录入即可;对于学院改名,把原学院对应的所有专业删除才能删除学院(防止误删学院),而后重新录入学院和对应的专业即可。
\item 竞赛管理,录入学校承认的比赛及其级别。
\item 教师信息管理,输入或删除建模的指导教师,包括工号。
\end{enumerate}

\begin{note}
对于改名等行为,我并没有做相关处理。如:对于信息工程学院而言,16,17,18级的学生是以信息工程学院的名称参赛的。而信息工程学院在2019下半年正式改名为人工智能学院,但在数据库的学生参赛纪录表中,学院名并不会更新,仍然保留信息工程学院,这是他们曾经存在过的印记。
\end{note}

\begin{note}
如果必须要更新,需要去数据库后台操作。防止操作失误,在本地模拟好and确认无误后,在后台执行。
\end{note}

\section{录入信息}

快速录入待开发,而手动录入的目的只是弥补快速录入中遗忘输入,或者修复错误输入的少部分数据。

手动输入数据的时候要注意,所有选项不能为空。

与录入相对应的就是删除信息,手动输入的时候难免会输入错误。这个时候不用慌,打开删除信息的界面,输入输入错误的学号进行查询,查询得到错误信息的唯一id,删除该id即可。

\section{查询信息}

\begin{enumerate}
    \item 普通游客输入学号和姓名就能查询,如果有重名现象,将全部显示。
    \item 超级管理员何以按照时间和比赛这两个选项导出excel格式的数据,之后怎么处理就看自己了。
\end{enumerate}

\section{建模博客}

我期待建模学子能搭建属于自己的博客,而不是沉迷在QQ空间或者微信朋友圈这样的交友范围受限、发布内容受限的平台。也不推荐CSDN、博客园和知乎等第三方平台,他们有严重的弊端,如:抄袭严重,排版混乱,广告泛滥,或者是人在美国、刚下飞机的吹嘘成风。(当然这些社区里面不乏有好文章,但比例不是很高,很多新手没有分辨文章好坏的能力)

搭建博客可以分享自己的旅行日志、下厨日志、技术文章、学习笔记和生活随笔等等等,而自己DIY的东西会更加爱护。能和更多人分享自己的经验心得,能遇见更广阔的天地和遇到更志同道合的人。这也是我开发这一选项的原因。

\section{个人空间}

预留的建模学子展示个人风采、组队的接口,但我不会开发这一项。所以只能预留这一个接口供后来者完善。

\end{document}