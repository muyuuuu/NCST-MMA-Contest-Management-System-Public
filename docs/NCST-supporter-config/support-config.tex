\documentclass[lang=cn,hazy,normal,blue,12pt]{elegantnote}
\setCJKmainfont{SourceHanSerifCN-Regular}

\title{华北理工大学数学建模协会竞赛管理软件\\——服务器端配置文档}

\author{16 计科刘佳玮}

\version{1.0}
\date{\today}

\begin{document}
\maketitle
\tableofcontents
\newpage

本文档的编写基于 Elegant \LaTeX{} Program。

\section{服务器购买}

首先,软件的运行模式为:用户运行软件后,软件自动联网读取网络中的数据库,通过点击软件的按钮等调用程序逻辑,程序逻辑连接网络、查询数据库将结果返回。因此,软件运行时必须联网,且要购买一个可靠的服务器。

\subsection{挑选服务器}

在购买服务器时,我们会遇到很多种类,如MySQL型、计算型和高IO型等。考虑到实际情况,软件的初始阶段访问量小,只需要一台低配置的云服务器实例即可运行应用程序。但随着网站发展,可以随时升级服务器实例的配置,或者增加服务器实例数量,无需担心低配计算单元在业务突增时带来的资源不足。

对于MySQL型的服务器,只能运行MySQL数据库,不能运行其他服务,如:部署python环境;对于高I/O型服务器,高速的I/O(数据读取)速度的确诱人,但价格要贵出两倍。考虑到软件运行期间不会有超大规模的访问量,所以暂时不考虑高I/O型服务器。

综上所述:数据库服务器只提供数据库服务,且连接到数据库服务器还需要其他服务器,成本较高;高I/O型数据库价格较贵;所以选择一般型的计算型服务器,可以运行python也可以部署服务器。

\subsection{挑选公司}

这里以2020年5月为例,在这之后的时间里市场行情可能会改变,所以这里写的并不一定准确,购买服务器时还请参考最新的行情标准。这里只介绍如何挑选各大公司的服务器,算是给一种通解吧,以阿里云和腾讯云为例。

首先看两公司在同一区间内的服务器配置,两者价格相差不到50元,所以,价格不能作为参考因素。

对于阿里云而言,阿里云的普通型计算服务器分分为突发型和共享型。共享型的意思是别人和你共享一套资源,即多人共用一套电脑,不可避免的会造成响应迟缓;突发型的意思是当你的需求突然增长时,服务器的性能也会突然增加,但性能提升上限会受到积分(阿里云的积分大概率需要充钱才能提升)的限制。所以,我没有选择阿里云。

腾讯云资源是独享的,即我一个人独享一个服务器,且性能不受到限制。而在仔细对比腾讯云和阿里云服务器的参数后,也是腾讯云的产品较为优良。最重要的一点是:我能看到腾讯服务器的详细配置清单,阿里云的却看不到(也可能是我没有找到),只是模糊的参数。有产品的详细描述则更加放心,腾讯云的产品介绍:\href{Details}{https://cloud.tencent.com/product\-/cvm/details}

\begin{note}
以上的对比情况只是我在2020年5月进行实地调查的结果,具体情况请根据当前年份的实际情况进行调整。
\end{note}

\subsection{参数选择与计费模式}

在购买腾讯云服务器的时候,会发现有服务器很多可供选择的参数,这里主要介绍一些很重要的参数选择标准。

\begin{enumerate}
    \item 操作系统:一定要选择Linux!!!Windows不够安全,可能会被黑客恶意攻击摧毁数据(目前服务器已经遭到了四次攻击)。腾讯云提供两种Liunx,分别为Centos和Ubuntu。这里建议选择Ubuntu,因为自带python。(腾讯云提供免费的上机,可以先试一下各大系统)
    \item CPU :CPU核心的数量表明了系统同一时间内最多能有几个进程在工作,在任务可以并行化的时候(如多人同时访问数据库),数量越多越好;CPU内存表示能有多大容量的数据直接和CPU交互,内存越大越好(脑补个人笔记本的内存)。当然,价格也会直线上升。个人的建议是,1核2GB是最低要求。如果后期软件的使用者越来越多,CPU 4核心8GB这个配置也是足够使用的。(4核8GB的价格为一年2500元,3年5000元)
    \item 地区:北京就好,学校离北京近,可以减少一些数据传输的时间。
    \item 数据盘:分为高性能云盘和SSD云盘。说人话的意思就是,高性能云盘位于网络各处,可能服务器在北京,云盘在上海;SSD云盘的意思是数据盘和服务器在一个地方,即服务器在北京,云盘就挨着服务器。显而易见,SSD的速度要快于高性能云盘,毕竟没有数据传输所需要耗费的时间。就价格而言,以腾讯云一年使用的使用期限为例,对于50GB的存储空间,SSD要比高性能云盘贵300块钱。
    \item 流量计费:众所周知,上网的流量是需要钱的。而腾讯云给了三种计费模式,分别为包年包月、按量计费和竞价实例。包年包月是云服务器实例一种预付费模式,提前一次性支付一个月或多个月甚至多年的费用;计费时间粒度精确到秒,不需要提前支付费用,每小时整点进行一次结算。此计费模式适用于电商抢购等设备需求量会瞬间大幅波动的场景,单价比包年包月计费模式高3到4倍;竞价实例是值可以以一定幅度的折扣购买实例,但同时系统可能会自动回收这些折扣售卖的实例。个人的建议是:按量计费太贵了,竞价实例不够稳定,还是直接包年或包月稳定一点。带宽为1Mbps/s时,一年的流量费为200元,相当于一个月17块钱的上网费。
\end{enumerate}

这里,给出我购买的参数供后来者参考:包年包月的流量计费;地区为北京;CPU为:标准型SA2.SMALL2,1核心2GB;系统为:Ubuntu 18.04;数据盘设置快照(免费);公网带宽1/Mbps;数据盘:50GB SSD 云硬盘;时间长度:3个月,价格为321元。

\begin{note}
购买时别选高级购买,选择自定义购买。自定义购买能看到更多可选择的参数。至于自定义购买时遇到不懂的参数,按它默认的来就好,如TCP协议端口等。
\end{note}

\begin{note}
一定要记住登录服务器的密码和服务器的公网IP地址。
\end{note}

\section{服务器端配置与软件安装}

在购买好服务器后,就进入了接下来的配置篇了,首先要完成的是服务器端的环境配置。一定要会一些Linux的命令行操作,一些文本编辑工具vim,nano等。Linux下没有可视化图形界面,不能用鼠标,全是小黑窗。不会打操作命令在Linux下寸步难行。

\begin{note}
这里的一些操作都是我边用边学边查的,不要畏首畏尾怕弄坏了,放心大胆的去做就好。
\end{note}

\subsection{服务器端配置}

检查有无python3和pip3,没有的话安装。

首先,安装Git这款软件,然后将软件的Github远程端的仓库克隆到本地(如何安装Git和Git的使用请自行搜索)。

然后安装MySQL,创建ncstmma这个数据库(记住这里的MySQL密码)。

最重要的一点是,将MySQL的编码修改为UTF-8。因为我们的数据库里面有中文,而默认的编码是不支持中文的。

执行models.py创建数据表。(models.py中最后的建表命令需要修改数据库的密码)

手动插入wuyuhang@admin超级管理员的账号和密码。(通俗一点,自己写MySQL语句插入)

\begin{note}
这里面的操作可能会比较繁琐,且细节较多,一定要按照靠谱文档操作。
\end{note}

\subsection{本地端配置}

本地软件就简单多了,将NcstModel.py中的服务器地址改为购买的服务器的公网IP地址(必须是公网),修改数据库的密码,修改登录服务器的密码,执行即可观察软件是否正常运行。如果软件运行没问题,最后打包成.exe文件发布即可。

\begin{note}
打包生成的软件可以直接运行,不是安装包。
\end{note}

\section{服务器迁移}

也许你会问,假设有一天我们换了新的服务器,如何把旧的服务器的数据迁移过去?抱歉我也不会,不过我相信网上会有很多答案。

\section{后续}

附一下我的学习流程吧,或者说,要开发这个软件或维护这个软件需要掌握哪些知识。

\begin{enumerate}
    \item Python: 掌握基本的用法就可以了,列表、字符和字典这种数据结构,能编写简单的程序。
    \item PyQt5:常用控件都学习过,每一个控件的常见效果需要了解。
    \item SqlAlchemy+Alembic:数据库的ORM,一定要会,不然服务器容易遭到攻击。
    \item MySQL:了解简单的查询语句就可以了。
    \item Linux:会常见的命令行操作,cd和ls等;必须会文本编辑工具nano,因为Linux没有图形化的界面供使用者操作。通俗一点,Linux下没办法运行word、记事本这样的软件,甚至鼠标都不能用。
    \item Git:软件的克隆,更新和版本控制的依赖工具。
    \item stackoverflow和google的使用,这两个网站可以解决99\%的代码错误问题,剩下的1\%是你的问题。
\end{enumerate}

当然,你也不必每个都学的很精通再来操作。就像建模一样,很多人第一次建模可以说啥都不会就参加了,实践出真知。不过我更喜欢有所准备的行动,至于没预料到的突发情况,随时查阅资料就好了。

\end{document}
